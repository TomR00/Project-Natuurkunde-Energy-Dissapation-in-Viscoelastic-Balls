\documentclass[11pt,a4paper]{article} 

\usepackage[dutch]{babel} %needs to specified for minutes package (else it will be in German)
\usepackage{a4wide}%For a wider spacing of the text (smaller left/right margin)

\usepackage{setspace}
\usepackage{minutes}

\pagestyle{plain}
\usepackage{todositemized}
%more memmorable commands to make the checked and crossed symbols





%%%%%%%%%%%%%%%%%%%%%%%%%%%%%%%%%%%%%%%%%%%%%%%%%%%%%%%%%%%%%%%%%%%%%%%%%%%%%%%
%
% Important: This template compiles without errors
% always check errors, also the yellow ones, even if you get a PDF
%
%%%%%%%%%%%%%%%%%%%%%%%%%%%%%%%%%%%%%%%%%%%%%%%%%%%%%%%%%%%%%%%%%%%%%%%%%%%%%%%%





\newpage



\usepackage[dutch]{babel} %needs to specified for minutes package (else it will be in German)
\usepackage{a4wide}%For a wider spacing of the text (smaller left/right margin)

\usepackage{setspace}
\usepackage{minutes}

\pagestyle{plain}
\usepackage{todositemized}
%more memmorable commands to make the checked and crossed symbols





%%%%%%%%%%%%%%%%%%%%%%%%%%%%%%%%%%%%%%%%%%%%%%%%%%%%%%%%%%%%%%%%%%%%%%%%%%%%%%%
%
% Important: This template compiles without errors
% always check errors, also the yellow ones, even if you get a PDF
%
%%%%%%%%%%%%%%%%%%%%%%%%%%%%%%%%%%%%%%%%%%%%%%%%%%%%%%%%%%%%%%%%%%%%%%%%%%%%%%%%



\begin{document}

\begin{Minutes}{Notulen Poject Natuurkunde, groepje 24}


%Add relevant date, time and location here
\minutesdate{13-06-2023} %Write the date of when you finish the minutes
\starttime{10:00}
\endtime{}
\location{}

%Add relevant names here
\participant{Tom, Noah, Talha} 
\minutetaker{Tom}

% \moderation{Niemand} 
%In case people are not present


\maketitle



\newpage


\section{Mededelingen} 
We begonnen vandaag om 10 uur. Talha en Tom hebben de metingen van de medium en soft putty met poeder afgerond. Daarna heeft Tom gewerkt aan de plot met de shift factor voor G''. Talha heeft verder gewerkt aan de plots voor alle verbanden tussen de meetwaarden. Noah heeft gewerkt aan de formules die Antoine had gegeven om te kijken of ze klopten. Ze hebben samen ook gebrainstormed en zijn met een nieuw research paper gekomen die hopelijk het verband tussen $\eta$ en $u_0$ beter beschrijft. Verder hebben Talha en Tom een begin gemaakt aan de poster en hebben ze twee slow-motion video's opgenomen die gebruikt kunnen worden bij de animatie. Verder heeft Antoine geprobeerd de rheometer op -30 graden te krijgen. Het is pas aan het eind van de dag rond 5 gelukt om hem aan de praat te krijgen.

\section{Wat moeten we nu doen/bespreken?}
We willen komende les hopelijk alle metingen met de rheometer te doen zodat we klaar zijn met het meten.


\end{Minutes}
\end{document}


\section{Oude actiepunten}
Zijn er niet.

\section{Wat moeten we nu doen/bespreken?}

\subsection{Wie wil en kan wat?}

\task{talha}{checklist}

\subsection{Communicatie}


\subsection{Beschikbaarheid}


\section{Checklist uit de Powerpoint}



\section{Overige punten}
We wachten even morgen af, als we met Antoine spreken.

\section{Nieuwe actiepunten}
\listoftasks
\section{Volgende vergadering}
De volgende bijeenkomst is morgen met de begeleider, bij het fancy koffiezet apparaat bij D.

\section{Afsluiting vergadering}
De vergadering is om 12:49 gesloten.


\end{Minutes}
\end{document}