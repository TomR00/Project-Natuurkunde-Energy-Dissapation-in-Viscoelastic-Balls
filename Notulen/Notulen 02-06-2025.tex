\documentclass[11pt,a4paper]{article} 

\usepackage[dutch]{babel} %needs to specified for minutes package (else it will be in German)
\usepackage{a4wide}%For a wider spacing of the text (smaller left/right margin)

\usepackage{setspace}
\usepackage{minutes}

\pagestyle{plain}
\usepackage{todositemized}
%more memmorable commands to make the checked and crossed symbols





%%%%%%%%%%%%%%%%%%%%%%%%%%%%%%%%%%%%%%%%%%%%%%%%%%%%%%%%%%%%%%%%%%%%%%%%%%%%%%%
%
% Important: This template compiles without errors
% always check errors, also the yellow ones, even if you get a PDF
%
%%%%%%%%%%%%%%%%%%%%%%%%%%%%%%%%%%%%%%%%%%%%%%%%%%%%%%%%%%%%%%%%%%%%%%%%%%%%%%%%



\begin{document}

\begin{Minutes}{Notulen Poject Natuurkunde, groepje 24}


%Add relevant date, time and location here
\minutesdate{02-06-2023} %Write the date of when you finish the minutes
\starttime{11:14}
\endtime{12:49}
\location{B1.19F}

%Add relevant names here
\participant{Talha, Tom, Noah} 
\minutetaker{Noah}

% \moderation{Niemand} 
\missingExcused{Madelief} %In case people are not present


\maketitle



\newpage


\section{Mededelingen} 
Dit is de eerste bijeenkomst. Noah notuleert nu, we kunnen een beetje doorwisselen. We wijzen geen voorzitter aan, maar Talha zal de agenda's opstellen vooraf aan de bijeenkomsten. We zullen via whatsapp communiceren wanneer we bijeen komen.

\section{Oude actiepunten}
Zijn er niet.

\section{Wat moeten we nu doen/bespreken?}
We nemen de communicatie handout voor ons, die nemen we door.

\subsection{Wie wil en kan wat?}
We gaan rond en iedereen geeft aan wat hij leuk of moeilijk vind. We merken dat we nog geen goed zicht hebben op hoe ons project er precies uit gaat zien, maar we hebben er wel zin in. Noah geeft aan dat hij al een keer bij de Softmatter group een Internship heeft gedaan, dit ging wel ergens anders over dan ons project.

\subsection{Teamverband}
Er lijkt een duidelijke verdeling tussen mensen die de theorie wat leuker vinden en het praktische deel leuker vinden, we denken ook op het zelfde niveau te zitten met programmeer skills. Het enige punt is dat 3 van de 4 mensen in ons groepje Dubba is en dus geen data-analyse hebben gehad. We gaan elkaar er doorheen helpen met hulp van Toms kennis van DAS. We zullen zodra het duidelijk is wat ons project precies zal omvatten een duidelijkere rolverdeling maken. We geven Tom de hoofdrol in het maken van de animatie. Uiteraard betekend dit niet dat hij het in zijn eentje gaat doen, maar hij zal hier een beetje de leiding/het overzicht over houden.

\subsection{Communicatie}
We hebben een whatsapp groep gemaakt, we communiceren via daar. Verder bestaat er deze overleaf, waar we aantekeningen, resultaten, etc. zullen bijhouden. We proberen zo veel mogelijk in het echt bijeen te komen, als het moet online. Mensen geven aan het liefst overdag (voor 5 uur) bijeen te komen/ te werken aan het project. Noah heeft een Nokia als je hem snel wilt bereiken, gebruik SMS. We zullen afspraken bijhouden in de omschrijving van de groepschat, hier zullen ook de actiepunten in komen. We willen een vast samenkom moment, maar dit is afhankelijk van wat de begleider van ons verwacht qua aanwezigheid etc. We wachten morgen even af.
\task{Talha}{Stel een vast samenkommoment vast zodra we met Antoine hebben gesproken}

Talha geeft aan het prettig te vinden als mensen hun deadlines gewoon halen. Geef het optijd aan als het niet lukt.

\subsection{Beschikbaarheid}
Iedereen is beschikbaar, maar men heeft ook hertentamens waar ze voor willen leren.

\section{Checklist uit de Powerpoint}
We gaan de lijst af uit de powerpoint.
\begin{itemize}
    \item Aanwezigheid: zie boven.
    \item We willen graag tenmiste een 6 halen, maar denken dat een 7 toch wel moet lukken.
    \item We werken graag overdag voor 5u, zodat we de avonden vrij houden. Ook niet te vroeg: vanaf 10u.
    \item We nemen de rubric door. Dingen die ons opvallen zijn de communicatie: moet onze conversatie via whatsapp ook ingeleverd worden? 
    \item Hoe doe je binnen 3 weken een poster-waardig onderzoek?
    \item Hoe ziet het practische deel van ons project eruit: in het lab, programmeren, iets anders?
    \item We hebben morgen met Antoine afgesproken, zie boven.
    \item Centrum van volume vs centrum of mass.
    \item We hebben allemaal enige ervaring met programmeren, maar zijn geen super ervaren programmeurs. Git hebben we allemaal nog nooit serieus gebruikt.
    \item 3 van de 4 mensen in ons groepje hebben geen DAS gehad. We hebben wel allemaal NSP1 gevolgd en gehaald. 
\end{itemize}



\section{Overige punten}
We wachten even morgen af, als we met Antoine spreken.

\section{Nieuwe actiepunten}
\listoftasks
\section{Volgende vergadering}
De volgende bijeenkomst is morgen met de begeleider, bij het fancy koffiezet apparaat bij D.

\section{Afsluiting vergadering}
De vergadering is om 12:49 gesloten.


\end{Minutes}
\end{document}