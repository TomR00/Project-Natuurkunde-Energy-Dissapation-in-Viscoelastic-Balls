\documentclass[11pt,a4paper]{article} 

\usepackage[dutch]{babel} %needs to specified for minutes package (else it will be in German)
\usepackage{a4wide}%For a wider spacing of the text (smaller left/right margin)

\usepackage{setspace}
\usepackage{minutes}

\pagestyle{plain}
\usepackage{todositemized}
%more memmorable commands to make the checked and crossed symbols





%%%%%%%%%%%%%%%%%%%%%%%%%%%%%%%%%%%%%%%%%%%%%%%%%%%%%%%%%%%%%%%%%%%%%%%%%%%%%%%
%
% Important: This template compiles without errors
% always check errors, also the yellow ones, even if you get a PDF
%
%%%%%%%%%%%%%%%%%%%%%%%%%%%%%%%%%%%%%%%%%%%%%%%%%%%%%%%%%%%%%%%%%%%%%%%%%%%%%%%%





\newpage



\usepackage[dutch]{babel} %needs to specified for minutes package (else it will be in German)
\usepackage{a4wide}%For a wider spacing of the text (smaller left/right margin)

\usepackage{setspace}
\usepackage{minutes}

\pagestyle{plain}
\usepackage{todositemized}
%more memmorable commands to make the checked and crossed symbols





%%%%%%%%%%%%%%%%%%%%%%%%%%%%%%%%%%%%%%%%%%%%%%%%%%%%%%%%%%%%%%%%%%%%%%%%%%%%%%%
%
% Important: This template compiles without errors
% always check errors, also the yellow ones, even if you get a PDF
%
%%%%%%%%%%%%%%%%%%%%%%%%%%%%%%%%%%%%%%%%%%%%%%%%%%%%%%%%%%%%%%%%%%%%%%%%%%%%%%%%



\begin{document}

\begin{Minutes}{Notulen Poject Natuurkunde, groepje 24}


%Add relevant date, time and location here
\minutesdate{10-06-2023} %Write the date of when you finish the minutes
\starttime{11:00}
\endtime{}
\location{}

%Add relevant names here
\participant{Tom, Noah, Talha} 
\minutetaker{Tom}

% \moderation{Niemand} 
%In case people are not present


\maketitle



\newpage


\section{Mededelingen} 
Vandaag begonnen we om 11 uur, de dag begon met Antione die ons weer wat stof uitlegde. In de middag hebben we vooral uitleg gekregen over hoe de rheometer werkt en hebben we hier een meting mee gedaan. Om 5 uur zijn we gestopt.

\section{Experiment}
De reometer is een apparaat die de viscositeit van de putty kan meten. Het is een plaat met een vooraf ingestelde temperatuur met een soort cylinder die omlaag komt, de putty indrukt en dan op verschillende frequenties gaat draaien om de viscositeit te maken als functie van de frequentie. We hebben de meting met medium putty gedaan. De plaat hebben we op 20 graden gezet. Dit is ongeveer kamertemperatuur, wat dezelfde temperatuur is waarmee we de videometingen hebben gedaan, dus moet de putty ongeveer dezelfde eigenschappen hebben.


\section{Wat moeten we nu doen/bespreken?}
We willen morgen nog meer metingen met de reometer gaan doen.

\section{Overige punten}
n.v.t

\end{Minutes}
\end{document}


\section{Oude actiepunten}
Zijn er niet.

\section{Wat moeten we nu doen/bespreken?}

\subsection{Wie wil en kan wat?}

\task{talha}{checklist}

\subsection{Communicatie}


\subsection{Beschikbaarheid}


\section{Checklist uit de Powerpoint}



\section{Overige punten}
We wachten even morgen af, als we met Antoine spreken.

\section{Nieuwe actiepunten}
\listoftasks
\section{Volgende vergadering}
De volgende bijeenkomst is morgen met de begeleider, bij het fancy koffiezet apparaat bij D.

\section{Afsluiting vergadering}
De vergadering is om 12:49 gesloten.


\end{Minutes}
\end{document}