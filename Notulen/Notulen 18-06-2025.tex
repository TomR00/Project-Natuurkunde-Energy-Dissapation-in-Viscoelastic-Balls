\documentclass[11pt,a4paper]{article} 

\usepackage[dutch]{babel} %needs to specified for minutes package (else it will be in German)
\usepackage{a4wide}%For a wider spacing of the text (smaller left/right margin)
\usepackage{graphicx}
\usepackage{setspace}
\usepackage{minutes}

\pagestyle{plain}
\usepackage{todositemized}
%more memmorable commands to make the checked and crossed symbols





%%%%%%%%%%%%%%%%%%%%%%%%%%%%%%%%%%%%%%%%%%%%%%%%%%%%%%%%%%%%%%%%%%%%%%%%%%%%%%%
%
% Important: This template compiles without errors
% always check errors, also the yellow ones, even if you get a PDF
%
%%%%%%%%%%%%%%%%%%%%%%%%%%%%%%%%%%%%%%%%%%%%%%%%%%%%%%%%%%%%%%%%%%%%%%%%%%%%%%%%





\newpage



\usepackage[dutch]{babel} %needs to specified for minutes package (else it will be in German)
\usepackage{a4wide}%For a wider spacing of the text (smaller left/right margin)

\usepackage{setspace}
\usepackage{minutes}

\pagestyle{plain}
\usepackage{todositemized}
%more memmorable commands to make the checked and crossed symbols





%%%%%%%%%%%%%%%%%%%%%%%%%%%%%%%%%%%%%%%%%%%%%%%%%%%%%%%%%%%%%%%%%%%%%%%%%%%%%%%
%
% Important: This template compiles without errors
% always check errors, also the yellow ones, even if you get a PDF
%
%%%%%%%%%%%%%%%%%%%%%%%%%%%%%%%%%%%%%%%%%%%%%%%%%%%%%%%%%%%%%%%%%%%%%%%%%%%%%%%%



\begin{document}

\begin{Minutes}{Notulen Poject Natuurkunde, groepje 24}


%Add relevant date, time and location here
\minutesdate{19-06-2023} %Write the date of when you finish the minutes
\starttime{11:00}
\endtime{}
\location{}

%Add relevant names here
\participant{Tom} 
\minutetaker{Tom}

% \moderation{Niemand} 
%In case people are not present


\maketitle



\newpage


\section{Mededelingen} 
Vandaag is Tom begonnen om 11:00. Hij heeft de -35 C in grafieken gezet om te kijken of deze ook overeenkomen met de data, dat deden ze inderdaad. Dit had hij een een Jupyter notebook gezet zodat we alle putty's in één code hadden maar wel los van elkaar. Verder kunnen we over de nieuwe data concluderen dat het gemodificeerde Hertz model klopt. Verder heeft hij tot 3 aan de poster gewerkt. Noah is een paar keer een minuutje komen kijken of het goed ging maar hij had wat anders die dag. Noah heeft rond 4 uur nog feedback over de poster aan Antoine gevraagd. Die feedback heeft Tom van 8 tot 9 verwerkt. Talha is absent in verband met overleden familielid.


\section{Wat moeten we nu doen/bespreken?}
Animatie maken en poster feedback verwerken.
(Tom en Talha gaan van het weekend aan de animatie werken, Noah kan het hele weekend niet)



\end{Minutes}
\end{document}




\section{Oude actiepunten}
Zijn er niet.

\section{Wat moeten we nu doen/bespreken?}

\subsection{Wie wil en kan wat?}

\task{talha}{checklist}

\subsection{Communicatie}


\subsection{Beschikbaarheid}


\section{Checklist uit de Powerpoint}



\section{Overige punten}
We wachten even morgen af, als we met Antoine spreken.

\section{Nieuwe actiepunten}
\listoftasks
\section{Volgende vergadering}
De volgende bijeenkomst is morgen met de begeleider, bij het fancy koffiezet apparaat bij D.

\section{Afsluiting vergadering}
De vergadering is om 12:49 gesloten.


\end{Minutes}
\end{document}