\documentclass[11pt,a4paper]{article} 

\usepackage[dutch]{babel} %needs to specified for minutes package (else it will be in German)
\usepackage{a4wide}%For a wider spacing of the text (smaller left/right margin)
\usepackage{graphicx}
\usepackage{setspace}
\usepackage{minutes}

\pagestyle{plain}
\usepackage{todositemized}
%more memmorable commands to make the checked and crossed symbols





%%%%%%%%%%%%%%%%%%%%%%%%%%%%%%%%%%%%%%%%%%%%%%%%%%%%%%%%%%%%%%%%%%%%%%%%%%%%%%%
%
% Important: This template compiles without errors
% always check errors, also the yellow ones, even if you get a PDF
%
%%%%%%%%%%%%%%%%%%%%%%%%%%%%%%%%%%%%%%%%%%%%%%%%%%%%%%%%%%%%%%%%%%%%%%%%%%%%%%%%





\newpage



\usepackage[dutch]{babel} %needs to specified for minutes package (else it will be in German)
\usepackage{a4wide}%For a wider spacing of the text (smaller left/right margin)

\usepackage{setspace}
\usepackage{minutes}

\pagestyle{plain}
\usepackage{todositemized}
%more memmorable commands to make the checked and crossed symbols





%%%%%%%%%%%%%%%%%%%%%%%%%%%%%%%%%%%%%%%%%%%%%%%%%%%%%%%%%%%%%%%%%%%%%%%%%%%%%%%
%
% Important: This template compiles without errors
% always check errors, also the yellow ones, even if you get a PDF
%
%%%%%%%%%%%%%%%%%%%%%%%%%%%%%%%%%%%%%%%%%%%%%%%%%%%%%%%%%%%%%%%%%%%%%%%%%%%%%%%%



\begin{document}

\begin{Minutes}{Notulen Poject Natuurkunde, groepje 24}


%Add relevant date, time and location here
\minutesdate{05-06-2023} %Write the date of when you finish the minutes
\starttime{11:00}
\endtime{}
\location{}

%Add relevant names here
\participant{Tom, Noah, Madelief} 
\minutetaker{Tom}

% \moderation{Niemand} 
%In case people are not present


\maketitle



\newpage


\section{Mededelingen} 
Vandaag begon we om 11 uur met wat uitleg over de theorie door Antione (supervisor) hij heeft ons uitleg gegeven over wat we precies kunnen doen met de verkregen data en hoe we dit het best kunnen visualiseren. Verder hebben we de laatste videometingen van de extra soft putty gebruikt om data te verkrijgen. Talha was helaas ziek vandaag. Rond 2 uur waren we hiermee klaar, Madelief is toen aan de notulen van gister gaan werken tot half 3 omdat ze toen weg moest, toen zijn Noah en Tom verder gegaan om nog een experiment te doen.

\section{Experiment}
We hebben in plaats van maizena gebruikgemaakt van magnesiumpoeder, dit wordt gebruikt bij klimmen. Wij hebben hiervoor gekozen omdat we merkte dat dit beter bleef zitten aan de buitenkant van de bal dan maizena en het redelijk makkelijk was aan te brengen. We probeerden zoveel mogelijk het poeder aan de buitenlaag te houden zodat er niet teveel poeder in de putty zelf gaat zitten wat de eigenschappen op langer termijn zou kunnen veranderen. Eerst hebben we een videometing gedaan met een stukje putty om te kijken of het niet bleef plakken, toen dit zo bleek hebben we daarna een aantal metingen gedaan met de extra soft putty omdat deze normaalgesproken het meest plakt. 


\section{Wat moeten we nu doen/bespreken?}
Morgen blijven we thuis, Antione heeft een lijstje gestuurd met dingen die dan gedaan kunnen worden, we moeten morgen via whatsapp communiceren wie wat gaat doen.

\section{Overige punten}
n.v.t

\end{Minutes}
\end{document}


\section{Oude actiepunten}
Zijn er niet.

\section{Wat moeten we nu doen/bespreken?}

\subsection{Wie wil en kan wat?}

\task{talha}{checklist}

\subsection{Communicatie}


\subsection{Beschikbaarheid}


\section{Checklist uit de Powerpoint}



\section{Overige punten}
We wachten even morgen af, als we met Antoine spreken.

\section{Nieuwe actiepunten}
\listoftasks
\section{Volgende vergadering}
De volgende bijeenkomst is morgen met de begeleider, bij het fancy koffiezet apparaat bij D.

\section{Afsluiting vergadering}
De vergadering is om 12:49 gesloten.


\end{Minutes}
\end{document}